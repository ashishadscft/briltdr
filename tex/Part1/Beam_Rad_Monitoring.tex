

\chapter{Beam and Radiation Monitoring Strategy}
\nte{this chapter is the overview where the various BRIL systems, their purpose and deliverables are briefly described - detailed technical descriptions should follow in Part II of the document}

\section{Safety and Beam Abort}
\nte{Describe tolerances from the tracker for damage.  What conditions could approach these tolerances. Describe a LHC failure scenario that the BCML would need to detect. Describe UFO (example from Florian thesis that went to 98 percent of abort)}

\section{Beam Induced Background Measurement}
\nte{Introduce the LHC simulations and beam background sources. From simulations, summarise LHC vs HL-LHC expectations for both sources.
Monitoring purpose for CMS. Describe the sources. Beam gas, near CMS. Tracker fluence and safety of the tracker, contribution to slow down track reconstruction (tbc).  Vacuum quality of the first triplet itself should also be monitored, in addition to the beam gas signatures in the instrumentation / cms sub-detectors. Strategy to measure with TEPX\_D4R1 and … 
Higher radius beam background, interactions with the TCT $>$ 150m from the IP. distance beam gas. Strategy to measure with BHM and ME4 and … 
Any operational aspects that affect CMs, e.g. contaminating the trigger rate / quality, track reconstruction, mis ID of missing MET etc.}

\section{Beam Timing}

\section{Radiation Monitoring}